\part{Gerenciamento de Requisitos}

\chapter[Gerenciamento de Requisitos]{Gerenciamento de Requisitos}


Gerenciar Requisitos é um processo associado à qualidade do desenvolvimento de software. Ela está tem a característica de ser um processo de entender e controlar diversas mudanças que ocorrem nos requisitos, por diversos motivos, como por exemplo mudanças no ambiente do sistema ou nos objetivos de uma organização [14], ele tem como principal objetivo a aquisição de conhecimentos das regras de negócios e verificação do que o cliente necessita para então obter uma boa especificação dos requisitos de software [13]. Para isso, neste tópico, foram estabelecidos alguns temas como a rastreabilidade dos requisitos e também os atributos dos requisitos que serão gerenciados.\\

\section{Rastreabilidade de Requisitos}
A rastreabilidade dos requisitos está diretamente ligada para referenciar um grupo coletivo de requisitos baseada em seus relacionamentos [15], ela estabelece formas de analisar o quanto mudanças afetaram o sistema.\\
\tab Os elementos para estabelecer os relacionamentos entre os artefatos de software e os requisitos são chamados de elos, estes são elementos que são necessários para estabelecer a Rastreabilidade [15] e a partir deles pode ser levada em consideração um aspecto fundamental para esse contexto: a habilidade de descrever a “vida” de um determinado requisito.\\
\tab O conceito de Rastreabilidade também pode ser definido como a capacidade de descrever e seguir o ciclo de vida de um requisito em diferentes direções [16]. Com isso os mais diversos requisitos e seus elos com determinados artefatos pode-se criar uma teia de relacionamentos em que a rastreabilidade tem como característica acompanhar justamente esses relacionamentos.\\
\tab Para este projeto foi escolhida a técnica vertical em que tem como característica relacionar artefatos dependendo de modelos. Para isso a rastreabilidade vertical será feita para Temas de Investimento, Épicos e Features e Casos de Uso.\\

\section{Atributos dos Requisitos}
Os requisitos do projeto irão ter alguns atributos que irão auxiliar no acompanhamento e gestão dos mesmos. Os atributos serão:\\
\tab \textbf{Data de criação do requisito:} quando o requisito foi criado na ferramenta;\\
\tab \textbf{Início previsto:} previsão de quando o requisito será desenvolvido;\\
\tab \textbf{Término Previsto:} previsão de quando irá terminar o desenvolvimento do requisito;\\
\tab \textbf{Data de início efetivo:} data em que foi iniciado o desenvolvimento do requisito;\\
\tab \textbf{Data de conclusão do requisito:} quando o requisito terminou de ser desenvolvido;\\
\tab \textbf{Valor para o negócio:} será classificada a relevância para o projeto o requisito em Alta, Média e Baixa:\\
\tab - Prioridade Alta: É um requisito fundamental para a solução, o não atendimento desse requisito não atende a necessidade do cliente.\\
\tab - Prioridade Média: É um requisito importante, porém a nível de satisfação do cliente mas não algo que é indispensável para a solução.\\
\tab - Prioridade Baixa: É um requisito que seria bom ter, mas não agrega valor a solução e pode ter seu desenvolvimento adiado.\\
\tab \textbf{Status do Requisito:} verifica a condição atual do requisito:\\
\tab - Temas de Investimento e Épicos: podem estar nos estados Novo, Em Progresso e Feito.\\
\tab - Features e Casos de Uso: podem estar nos estados Aberto, Em Progresso, Em Teste e Feito.\\
\tab \textbf{Esforço:} Será uma característica que irá indicar o quanto um requisito toma de esforço para ser concluído. Para isso será utilizado o Planning Poker, em que os envolvidos devem pontuar o quanto toma de esforço um requisito até chegarem a um consenso.\\
