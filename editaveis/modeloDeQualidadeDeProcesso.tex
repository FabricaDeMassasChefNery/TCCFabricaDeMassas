\part{Modelo de Qualidade de Processo}
\chapter[Modelo de Qualidade de Processo]{Modelo de Qualidade de Processo}
{\large {\section { Modelo de Qualidade de processo \\ } } }
{MPS.BR (Melhoria de Processos do Software Brasileiro) é um modelo hierárquico de qualidade de
processos de software. Cada nível, sendo G o mais baixo e A o mais alto, agrega práticas de processos que
melhoram a qualidade da produção de software. (SOFTEX, 2013).\\
Os processos de requisitos aparecem nos níveis G e D, os quais serão descritos a seguir:\\
}
{\large Práticas implementadas no projeto:\\}
{\footnotesize {\textbf { *Para detalhes sobre as atividades mencionadas, ver tabelas na seção Atividades. \\ }}}

\begin {enumerate}
{\item {\large \textbf{Nível G - Parcialmente Gerenciado \\}}}

\begin  {itemize}
{\item {Gerência de Requisitos (GRE)}}

{\tab Propósito: \textit{gerenciar os requisitos do produto e dos componentes do produto do projeto e identificar inconsistências entre os requisitos, os planos do projeto e os produtos de trabalho do projeto.\\}}

{Resultados esperados:\\}

{\item \textbf{GRE 1. (implementado)} O entendimento dos requisitos é obtido junto aos fornecedores de requisitos.}

Implementado na seção de Portfólio do processo de Requisitos, através das atividades Compreender o contexto e os objetivos do cliente (PO01), Definir Épicos (PO02) e Priorizar Épicos (PO05).
Também implementado na seção de Programa do processo de Requisitos, através das atividades Definir Features(PR03) e Priorizar Features(PR06).
Também implementado na seção de Time do processo de Requisitos, através das atividades Definir Histórias do Usuário (T01) e Priorizar Histórias do Usuário (T03).

{\item \textbf{GRE 2. (implementado)} Os requisitos são avaliados com base em critérios objetivos e um comprometimento da equipe técnica com estes requisitos é obtido;}

Implementado na seção de Portfólio do processo de Requisitos, através das atividades Definir épicos (PO02), Definir enables para os épicos (PO3) e Priorizar épicos (PO04).
Também implementado na seção de Programa do processo de Requisitos, através das atividades Elaborar Visão (PR01), Definir features (PR03), Definir requisitos não-funcionais(PR04), Definir enables para features (PR05) e Priorizar features (PR06).
Também implementado na seção de Time do processo de Requisitos, através da atividade Definir histórias de usuário (T01) e Priorizar histórias de usuário (T03).

{\item \textbf{GRE 3. A rastreabilidade bidirecional entre os requisitos e os produtos de trabalho é estabelecida e mantida;}}

{\item \textbf{GRE 4. (implementado) Revisões em planos e produtos de trabalho do projeto são realizadas visando identificar e corrigir inconsistências em relação aos requisitos;}}

Implementado na seção de Portfólio do processo de Requisitos, através da atividade Validar Épicos (PO05). Como é possível ver no diagrama do processo, a continuação do processo após a PO05, que essencialmente é uma atividade de revisão,  é condicionada por seu resultado. Caso a validação seja positiva, prossegue-se com o fluxo normal do processo; caso contrário, o processo retorna à estágios anteriores, para correção.
Também implementado na seção de Programa do processo de Requisitos, através da atividade Validar features (PR08). A continuação do processo após a atividade PR08 é condicionada de forma análoga a da atividade PO05, citada acima, em função de seu resultado.

{\item \textbf{GRE 5. Mudanças nos requisitos são gerenciadas ao longo do projeto.}}

Implementado nas seções de Programa e Time do processo de Requisitos, através da atividade Gerenciar requisitos (PR02 e T02).

{\item {\large \textbf{Nível D - Largamente definido:}}}

{\item {Desenvolvimento de Requisitos (DRE)}}

{\tab Propósito: \textit{Propósito: definir os requisitos do cliente, do produto e dos componentes do produto.}}

Resultados esperados:

{\item \textbf{DRE 1. As necessidades, expectativas e restrições do cliente, tanto do produto quanto de suas interfaces, são identificadas;}}

Implementado na seção de Portfólio do processo de Requisitos, através da atividade Compreender o contexto e os objetivos do cliente (PO01), Definir Épicos (PO02) e Priorizar Épicos (PO04).
Também implementado na seção de Programa do processo de Requisitos, através das atividades Definir Features(PR03) e Priorizar Features(PR06).
Também implementado na seção de Time do processo de Requisitos, através das atividades Definir Histórias do Usuário (T01) e Priorizar Histórias do Usuário (T03).

{\item \textbf{DRE 2. Um conjunto definido de requisitos do cliente é especificado e priorizado a partir das necessidades, expectativas e restrições identificadas;}}

Implementado na seção de Portfólio do processo de Requisitos, através da atividade Priorizar Épicos (PO05).
Também implementado na seção de Programa do processo de Requisitos, através da atividade Priorizar features (PR06).
Também implementado na seção de Time do processo de Requisitos, através das atividades Definir Histórias do Usuário (T01) e Priorizar Histórias do Usuário (T03).

{\item \textbf{DRE 3. Um conjunto de requisitos funcionais e não-funcionais, do produto e dos componentes do produto que descrevem a solução do problema a ser resolvido, é definido e mantido a partir dos requisitos do cliente;}}

As atividades citadas no DRE 2, conjuntamente com a atividade Elaborar Requisitos não-funcionais (PR04) são responsáveis pela geração dos conjuntos de requisitos funcionais e não-funcionais, e a atividade Gerenciar requisitos (PR02 e T02) as mantém.

{\item \textbf{DRE 4. Os requisitos funcionais e não-funcionais de cada componente do produto são refinados, elaborados e alocados;}}

Implementado na seção de Portfólio do processo de Requisitos, através da atividade Priorizar Épicos (PO04).
Também implementado na seção de Programa do processo de Requisitos, através das atividades Elaborar Visão (PR01), Priorizar features (PR06), Elaborar Roadmap (PR07) e Elaborar PI (PR09).
Também implementado na seção de Time do processo de Requisitos, através das atividades Priorizar Histórias do Usuário (T03) e Sprint Planning (T04).

{\item \textbf{DRE 5. Interfaces internas e externas do produto e de cada componente do produto são definidas;}}

Interfaces do produto são definidas nas atividades mencionadas em DRE1. Interfaces das componentes do produto não são definidas; o pequeno tamanho do sistema demanda pouca necessidade de tal atividade e portanto ela será ignorada.

{\item \textbf{DRE 6. Conceitos operacionais e cenários são desenvolvidos;}}

Conceitos operacionais e cenários não serão desenvolvidos. O nível de formalidade do projeto demanda pouca necessidade de tais atividades e portanto elas serão ignoradas.

{\item \textbf{DRE 7. Os requisitos são analisados, usando critérios definidos, para balancear as necessidades dos interessados com as restrições existentes;}}

Não foram definidos critérios para tal atividade.

{\item \textbf{DRE 8. Os requisitos são validados.}}

Implementado na seção de Portfólio do processo de Requisitos, através da atividade Validar Épicos (PO05).
Também implementado na seção de Programa do processo de Requisitos, através das atividades Validar Features(PR08).

fonte: SOFTEX, Guia Geral MPS de Software.

\end {itemize}
\end {enumerate}
