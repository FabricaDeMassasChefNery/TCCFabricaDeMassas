

\chapter[Elicitação de Requisitos]{Elicitação de Requisitos}

Uma Elicitação de requisitos tem por característica achar o que realmente o usuário precisa para uma futura implementação de um software, é um processo que define o que o software deve ter como funcionalidade. Elicitar também pode ser descrito como descobrir ou obter informação sobre algo. O sucesso de um sistema de informação depende da qualidade da definição dos requisitos (CRISTIANO, 1994). A isso se deve a grande importância de fazer um bom levantamento de requisitos.\\
\tab Diversos projetos de software possuem a característica de terem falhado por problemas de Elicitação de Requisitos (BOEHM, 1981) (GOGUEN, 1997). Pode ser entendido que muitas vezes, os requisitos são mal interpretados ou incompletos. Com uma elicitação mal feita estariam comprometidos tanto a documentação quanto a especificação dos requisitos, provocando a inviabilidade de todo processo de Engenharia de Requisitos.\\
\tab O processo de Engenharia de requisitos é a primeira etapa dentro de todo o processo de Engenharia de Requisitos (THAYER, 1997). Nela, a principal preocupação é em como levantar os reais requisitos do sistema. Para isso, existem diversas maneiras de levantar requisitos. \\

\section{Técnicas de Elicitação de Requisitos}
\tab Foram estabelecidos critérios para a seleção das técnicas de elicitação de requisitos usadas. Estes foram:\\ \\
\tab - Disponibilidade da Equipe;\\
\tab - Disponibilidade do Cliente;\\
\tab - Compatibilidade das Técnicas com a abordagem.\\ \\
\tab A partir desses critérios serão/foram utilizadas as seguintes técnicas:\\

\subsection{Observação}
\tab Essa técnica tem por característica possibilitar um contato pessoal estreito do pesquisador com o fenômeno pesquisado (BELGAMO, 2000), no caso do presente projeto, um dos integrantes tem ligação parentesca com o cliente, isso facilita muito tanto a comunicação quanto o entendimento do trabalho para todo o grupo.\\

\subsection{Entrevista}
\tab Essa é uma técnica que tem como base o engenheiro ou analista discutir o sistema com diferentes usuários, a partir disso elabora um entendimento dos requisitos (BELGAMO, 2000). No caso foi feita uma entrevista aberta, em que houve uma conversa para determinar o que o cliente necessita a partir de seus problemas.\\

\subsection{Protótipo}
\tab É um sistema sem funcionalidades inteligentes. O protótipo cria uma interface para o cliente validar o software que será implementado.\\

\subsection{Workshop}
\tab É uma técnica tem um objetivo pré determinado onde cada indivíduo tem direito a fala e são discutidos os requisitos de acordo com esse objetivo. Em um Workshop há um indivíduo que é responsável por conduzir a reunião e gerar discussões entre os envolvidos.\\

