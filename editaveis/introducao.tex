\chapter*[Introdução]{Introdução}
\addcontentsline{toc}{chapter}{Introdução}

O presente trabalho tem como finalidade apresentar o resultado da Processo de Engenharia de Requisitos de Software para a Fábrica de Massas do Chef Nery. O empreendimento trata-se de uma micro empresa que fabrica diferentes tipos de massas.\\
\tab O contexto relacionado a necessidade do projeto é principalmente relacionado a necessidade de um melhor gerenciamento de clientes visando então buscar uma melhor forma de gerenciar seus pedidos e divulgar seus produtos.\\
\tab Para isso, é de fundamental importancia utilizar uma metodologia para o levantamento de requisitos e gerenciamento do processo em questão.\\
\tab Com uma análise acerca do contexto foi utilizado uma abordagem ágil composto de atividades do Scaled Agile Framework (SAFe). Com isso, foi possível desenhar o Processo de Engenharia de Requisitos contendo um também um modelo do Processo que será implementado no Trabalho 2.\\

{\large{1.1 Visão Geral do Relatório}}\\ \\
\tab - Introdução: tem como finalidade apresentar um resumo deste documento;\\
\tab - Contexto da Empresa (Chef Nery): fundamenta o contexto da empresa contendo;\\
\tab - Justificativa da Abordagem: tem como finalidade explicar o motivo da adoção da abordagem;\\
\tab - Processo de Engenharia de Requisitos: tem como finalidade apresentar o Processo de Engenharia de Requisitos de Software para a Fábrica de Massas;\\
\tab - Elicitação de Requisitos: apresenta como serão feitas as elicitações dos requisitos;\\
\tab - Tópicos de Gerenciamento de Requisitos: tem como finalidade mostrar como serão aplicadas as praticas de rastreabilidade de requisitos e os atributos que serão utilizados para os mesmos;\\
\tab - Planejamento do Projeto;\\
\tab - Ferramentas de Gestão de Requisitos: apresenta ferramentas de gerencia de requisitos e as analisa de acordo com características;\\
\tab - Considerações Finais;\\
\tab - Referências;\\
