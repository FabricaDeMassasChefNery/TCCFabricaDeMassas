\chapter[Processo de Engenharia de Requisitos]{Processo de Engenharia de Requisitos}
{
	\large{Processo de Engenharia de Requisitos\\}

}


\begin{figure}[h]
    \centering
    \label{fig01}
        \includegraphics[keepaspectratio=true,scale=0.5]{figuras/ProcessoSAFE69.eps}
    \caption{O Processo}
\end{figure}


{
	\large{\section {Níveis do Processo} }

	\tab A seguir serão apresentados os níveis de abstração do processo de Engenharia de Requisitos. Ressalta-se o fato que este processo foi baseado no SAFe 4.0. A seguir serão apresentados estes níveis com maior detalhamento\\

}

{
 \large{\subsection {Nível de Portfólio} }
	\tab O nível de Portfólio é o nível de abstração mais alto do SAFe. Nesse nível são providos as estruturas necessárias para o desenvolvimento do projeto.  Nele são desenvolvidos: Enables de Épicos; Temas de Investimentos; e os Épicos. \\

}

{
	 \large{\subsection {Nível de Programa} }

	 \tab O nível de Programa é o intermédio entre o nível de programa e o do Time. Desta forma, neste nível os times de desenvolvimento e outros recursos são aplicados na tarefa de desenvolvimento do projeto.  Nesta etapa ocorre o refinamento dos Épicos que se transformam em Features e o planejamento do Program Increment (PI).\\
}

{
	\large{\subsection {Nível de Time} }

	\tab Este nível se assemelha bastante ao SCRUM,  portanto existem algumas atividades, tais como : Sprint planning, sprint review, sprint retrospective. Nesta etapa, as Features são fragmentadas em Histórias do Usuário. Esta etapa está relacionada com o desenvolvimento dos PI’s. \\
}

{
	\large{\section {Papéis do Processo} }

	\tab A seguir serão apresentados os papéis adotados para o processo de Engenharia de Requisitos da Fábrica de massas ("Chef Nery"). Ressalta-se o fato que este processo foi baseado no SAFe 4.0. A seguir serão apresentados os papéis do processo.\\

}


{\large{\subsection {Papéis do Nível de Portfólio} } }
\begin{itemize}
{
	\large{\item {Gerente de Portfólio\\} }

	\tab O Gerente de Portfólio tem como responsabilidade elaborar e comunicar temas estratégicos que implicam nas estratégias da empresa. Ressalta-se também o fato de ser responsável pela elaboração e priorização dos Épicos. \\
	\tab Em nosso contexto, o integrante Eduardo Gomes será o responsável por este papel. \\
}


{
	\large{\item {Arquiteto do Sistema\\} }

	\tab O Arquiteto de Sistemas tem a responsabilidade de manter as soluções da empresa de forma holística e iniciativas de desenvolvimento. Além disso, deve fornecer caminhos arquiteturais da empresa que possibilite o suporte aos Enables de Épicos. \\
	\tab Neste contexto, o integrante Miguel Pimentel será o responsável por este papel. \\
}

{\large{\subsection {Papéis do Nível de Programa} } }

{
	\large{\item {Gerente do Produto\\} }

	\tab O gerente do Produto (Product Manager)  tem como responsabilidade: compreender e entender as necessidades do cliente; validar as soluções desenvolvidas para a necessidade do cliente;  compreender e auxiliar as atividades do nível de Portfólio. Priorizar o fluxo de trabalho do PI; desenvolver Visão; e gerenciar Release. \\
	\tab Em nosso contexto, o papel de gerente do produto será empenhado pelo integrante Miguel Nery. \\
 }

{
\large{\item {Scrum Master\\} }
	\tab O Scrum Master é responsável por asseguram as práticas e princípios do SCRUM e XP. Este papel também deve manter o foco do time, e assim, facilitar o progresso do time. \\
	\tab Em nosso contexto, O Scrum Master será um papel rotativo entre todos os integrantes do time. \\
}

{
	\large{\item {Product Owner\\} }

	\tab O Product Owner (PO) é o membro do time responsável pela definição das histórias e priorização do Backlog tanto como simplificar as prioridades do programa. \\
	\tab O PO também pode realizar algumas tarefas conhecidas, tais como: desenvolvimento e participação no PI planning; definir metas da sprint; definir prioridades  a partcir do backlog; execução do programa; manter o programa; trabalhar em harmonia com o Product Manager; participar do planejamento e validação da sprint. \\
	\tab Em nosso contexto, PO será empenhado por Pedro Nery (Dono da fábrica de massas “Chef Nery”). \\
}

{
	\large{\item {Desenvolvedores (Time)\\} }

	\tab O time também conhecido como Equipe de Desenvolvedores está relacionado com o terceiro nível de abstração. Além disso, é responsável pela pela entrega de uma versão usável de cada incremento provido de um PI. \\
	\tab Em nosso contexto, todos os integrantes da equipe realizaram o papel de desenvolvedores. \\

}

{
	\large{\section {Artefatos do Processo } }

	\tab A seguir serão apresentados os artefatos oriundos do processo de obtenção de Requisitos da Fábrica de massas ("Chef Nery"). Ressalta-se o fato que este processo foi baseado no SAFe 4.0. A seguir serão apresentados os artefatos como uma breve descrição destes. \\

}

{\large{\subsection{Artefatos em Nível de Portfólio }}}

{
	\large{\item Temas de Investimento \\}

	\tab Temas estratégico é artefato responsável por todas as questões de negócio do projeto. Este artefato é relacionado com o nível de abstração Portfólio. \\
}


{
	\large{\item Backlog de  Épicos \\}

	\tab O Backlog de Épicos é um artefato que engloba os épicos que foram levantados juntos a cliente e serão produzidos. Vale ressaltar que eles possuem um priorização. \\
}

{\large{\subsection {Artefatos em Nível de Programa }}}

{
	\large{\item Visão \\}

	\tab A artefato de visão descreve uma noção das soluções a serem desenvolvidas, refletindo as necessidades do cliente e stakeholders, como também as features e soluções propostas a estas necessidades. \\
}


{
	\large{\item Backlog de Features \\}

	\tab Apresenta o conjunto de features que serão implementadas para o sistema, estas devem estar pontuadas de acordo com a sua dificuldade técnica. Pode-se utilizar ferramentas como o planning poker para a avaliação da dificuldade técnica das features levantadas. \\
}

{
	\large{\item Roadmap \\}

	\tab Apresenta os entregáveis próximos a partir de uma linha do tempo, em outras palavras, apresenta as features organizadas de acordo com a priorização definida previamente. \\
}

{
	\large{\item Especificação Suplementar\\}

	\tab Este artefato apresenta todas as definições necessárias do negócio não incluídas na definição de Requisitos funcionais, no caso, Backlog de Features e Backlog do Time. \\
}

{\large{\subsection{Artefatos em Nível de Time }}}

{
	\large{\item Backlog do Time \\}

	\tab Representa um conjunto de histórias priorizada de acordo com os interesses do cliente. Elas consistem da fragmentação e refinamento das features definidas no nível do programa. Além disso, o Backlog do time também pode ser entendido como o conjunto de todas as coisas que o time deve fazer. \\
}

{
	\large{\item Sprint Backlog\\}

	\tab Consiste em todas as histórias de usuário que devem ser desenvolvidas durante uma sprint. Vale ressaltar que estas histórias são oriundas do Backlog do Time. \\
}
\end{itemize}
{

	\large{\subsection {Atividades do Processo}}

	\tab As atividades  estão dispostas a seguir de acordo com o quadro apresentado abaixo. Não obstante, cada atividade se encontra no subtópico referente ao seu nível de abstração. \\
	\tab No quadro abaixo, o identificador foi criado para identificar a atividade de acordo com o seu nível de abstração. Desta forma, os identificadores estão representados da seguinte maneira:

	\begin{itemize}
		\item Atividades que começarem com as iniciais PO, referem-se ao nível de abstração Portfólio e o número da atividade neste nível. Exemplo: PO01 (Atividade 1 do nível de abstração Portfólio);
		\item Atividades que começarem com as iniciais PR, referem-se ao nível de abstração Programa e o número da atividade neste nível. Exemplo: PR01(Atividade 1 do nível de abstração Programa);
		\item Atividades que começarem com as iniciais T, referem-se ao nível de abstração Time e o número da atividade neste nível. Exemplo: T03(Atividade 3 do nível de abstração Time);
	\end{itemize}

	\tab Além disso, para atividades que não são necessárias entradas ou possuem não apresentam nenhuma saída será identificado pela sigla N/A (Não se aplica).\\


	\begin{table}[H]
                \centering
                \caption{Modelo das atividades}
                \begin{tabular}{c|p{10cm}}
                    \hline
                    \textbf{Nome}            & Nome da atividade coeso com a atividade que deve ser realizada\\
                    \hline
                    \textbf{Identificador} & Identifica a atividade conforme foi descrito anteriormente\\
                    \hline
                    \textbf{Descrição}   & Breve descrição da atividade\\
                    \hline
                    \textbf{Entrada}           & Indica qual dos artefatos serão utilizados na realização desta atividade\\
                    \hline
                    \textbf{Saída}            &  Indica os artefatos que foram elaborados ao término desta atividade\\
                    \hline
                    \textbf{Responsável}            & Indica qual dois papéis é responsável pela execução  desta atividade\\
                    \hline
                \end{tabular}
            \end{table}


    \large{Atividades do Nível de Portfólio}



          \begin{table}[H]
                \centering
                \caption{Compreender o contexto e os objetivos do cliente}
                \begin{tabular}{c|p{10cm}}
                    \hline
                    \textbf{Nome}            & Compreender o contexto e os objetivos do cliente\\
                    \hline
                    \textbf{Identificador} & PO01\\
                    \hline
                    \textbf{Descrição}   & Compreender quais são as necessidades do cliente  a partir do contexto que ele está inserido\\
                    \hline
                    \textbf{Entrada}           & N/A\\
                    \hline
                    \textbf{Saída}            &  Temas de Investimento\\
                    \hline
                    \textbf{Responsável}            & Gerente de Portfólio\\
                    \hline
                \end{tabular}
            \end{table}


          \begin{table}[H]
                \centering
                \caption{Definir Épicos}
                \begin{tabular}{c|p{10cm}}
                    \hline
                    \textbf{Nome}            & Definir Épicos\\
                    \hline
                    \textbf{Identificador} & PO02\\
                    \hline
                    \textbf{Descrição}   & Definir os Épicos de acordo com as necessidade e contexto do cliente\\
                    \hline
                    \textbf{Entrada}           & Temas de Investimento\\
                    \hline
                    \textbf{Saída}            &  Épicos\\
                    \hline
                    \textbf{Responsável}            & Gerente de Portfólio\\
                    \hline
                \end{tabular}
            \end{table}


		 \begin{table}[H]
                \centering
                \caption{Definir Enables para os Épicos}
                \begin{tabular}{c|p{10cm}}
                    \hline
                    \textbf{Nome}            & Definir Enables para os Épicos\\
                    \hline
                    \textbf{Identificador} & PO03\\
                    \hline
                    \textbf{Descrição}   & Definir quais serão os enables para os épicos previamente definidos\\
                    \hline
                    \textbf{Entrada}           & Épicos\\
                    \hline
                    \textbf{Saída}            &  Enables de Épicos\\
                    \hline
                    \textbf{Responsável}            & Gerente de Portfólio e Arquiteto do Sistema\\
                    \hline
                \end{tabular}
            \end{table}

             \begin{table}[H]
                \centering
                \caption{Priorizar Épicos}
                \begin{tabular}{c|p{10cm}}
                    \hline
                    \textbf{Nome}            & Priorizar Épicos\\
                    \hline
                    \textbf{Identificador} & PO04\\
                    \hline
                    \textbf{Descrição}   & Definir quais épicos devem possuir maior prioridade sobre os outros\\
                    \hline
                    \textbf{Entrada}           & Enables de Épicos e Épicos\\
                    \hline
                    \textbf{Saída}            &  Backlog de Épicos\\
                    \hline
                    \textbf{Responsável}            &  Gerente de Portfólio\\
                    \hline
                \end{tabular}
            \end{table}

            \begin{table}[H]
                \centering
                \caption{Validar Épicos}
                \begin{tabular}{c|p{10cm}}
                    \hline
                    \textbf{Nome}            & Validar Épicos\\
                    \hline
                    \textbf{Identificador} & PO05\\
                    \hline
                    \textbf{Descrição}   & Validar junto ao cliente se os Épicos atendem suas necessidade\\
                    \hline
                    \textbf{Entrada}           & Backlog de Épicos\\
                    \hline
                    \textbf{Saída}            &  N/A\\
                    \hline
                    \textbf{Responsável}            & Product Owner e  Gerente de Produto\\
                    \hline
                \end{tabular}
            \end{table}

    \large{Atividades do Programa\\}

            \begin{table}[H]
                \centering
                \caption{Elaborar Visão}
                \begin{tabular}{c|p{10cm}}
                    \hline
                    \textbf{Nome}            & Elaborar Visão\\
                    \hline
                    \textbf{Identificador} & PR01\\
                    \hline
                    \textbf{Descrição}   & Desenvolver o artefato de visão\\
                    \hline
                    \textbf{Entrada}           & Backlog de Épicos e Temas de Investimento\\
                    \hline
                    \textbf{Saída}            &  Visão\\
                    \hline
                    \textbf{Responsável}            & Arquiteto do Sistema e Gerente do Produto\\
                    \hline
                \end{tabular}
            \end{table}

            \begin{table}[H]
                \centering
                \caption{Gerenciar Requisitos}
                \begin{tabular}{c|p{10cm}}
                    \hline
                    \textbf{Nome}            & Gerenciar Requisitos\\
                    \hline
                    \textbf{Identificador} & PR02\\
                    \hline
                    \textbf{Descrição}   & Identificar, avaliar e acordar modificações nos requisitos\\
                    \hline
                    \textbf{Entrada}           & Temas de Investimento e Épicos \\
                    \hline
                    \textbf{Saída}            &  Atualização dos requisitos\\
                    \hline
                    \textbf{Responsável}            & Gerente do Produto\\
                    \hline
                \end{tabular}
            \end{table}



              \begin{table}[H]
                \centering
                \caption{Definir Features}
                \begin{tabular}{c|p{10cm}}
                    \hline
                    \textbf{Nome}            & Definir Features\\
                    \hline
                    \textbf{Identificador} & PR03\\
                    \hline
                    \textbf{Descrição}   & Refinar os Épicos com o objetivo de obter features\\
                    \hline
                    \textbf{Entrada}           & N/A\\
                    \hline
                    \textbf{Saída}            &  Features\\
                    \hline
                    \textbf{Responsável}            & PO, Arquiteto de Sistema e Gerente do Produto\\
                    \hline
                \end{tabular}
            \end{table}

              \begin{table}[H]
                \centering
                \caption{Elaborar Requisitos Não-Funcionais}
                \begin{tabular}{c|p{10cm}}
                    \hline
                    \textbf{Nome}            & Elaborar Requisitos Não-Funcionais\\
                    \hline
                    \textbf{Identificador} & PR04\\
                    \hline
                    \textbf{Descrição}   & Definir características e necessidades do sistema que não seja funcionalidades\\
                    \hline
                    \textbf{Entrada}           & N/A\\
                    \hline
                    \textbf{Saída}            &  Especificação Suplementar\\
                    \hline
                    \textbf{Responsável}            & Gerente do Produto, Arquiteto do Sistema e PO\\
                    \hline
                \end{tabular}
            \end{table}

              \begin{table}[H]
                \centering
                \caption{Definir Enables para Features}
                \begin{tabular}{c|p{10cm}}
                    \hline
                    \textbf{Nome}            & Definir Enables para Features\\
                    \hline
                    \textbf{Identificador} & PR05\\
                    \hline
                    \textbf{Descrição}   & Definir recursos que permitam a execução das features\\
                    \hline
                    \textbf{Entrada}           & Features\\
                    \hline
                    \textbf{Saída}            &  Enables para Features\\
                    \hline
                    \textbf{Responsável}            & Arquiteto do Sistema e Gerente do Produto\\
                    \hline
                \end{tabular}
            \end{table}

              \begin{table}[H]
                \centering
                \caption{Priorizar Features}
                \begin{tabular}{c|p{10cm}}
                    \hline
                    \textbf{Nome}            & Priorizar Features\\
                    \hline
                    \textbf{Identificador} & PR06\\
                    \hline
                    \textbf{Descrição}   & Definir quais features devem possuir maior prioridade sobre os outros\\
                    \hline
                    \textbf{Entrada}           & Features\\
                    \hline
                    \textbf{Saída}            &  Backlog de Features\\
                    \hline
                    \textbf{Responsável}            & Gerente do Produto e PO\\
                    \hline
                \end{tabular}
            \end{table}

              \begin{table}[H]
                \centering
                \caption{Elaborar Roadmap}
                \begin{tabular}{c|p{10cm}}
                    \hline
                    \textbf{Nome}            & Elaborar Roadmap\\
                    \hline
                    \textbf{Identificador} & PR07\\
                    \hline
                    \textbf{Descrição}   & Elaborar o artefato Roadmap\\
                    \hline
                    \textbf{Entrada}           & Visão e Backlog das Features\\
                    \hline
                    \textbf{Saída}            &  Roadmap\\
                    \hline
                    \textbf{Responsável}            & Arquiteto do Sistema e Gerente do Produto\\
                    \hline
                \end{tabular}
            \end{table}

              \begin{table}[H]
                \centering
                \caption{Validar Features}
                \begin{tabular}{c|p{10cm}}
                    \hline
                    \textbf{Nome}            & Validar Features\\
                    \hline
                    \textbf{Identificador} & PR08\\
                    \hline
                    \textbf{Descrição}   & Validar junto ao cliente se as features atendem suas necessidades\\
                    \hline
                    \textbf{Entrada}           & Backlog de Features\\
                    \hline
                    \textbf{Saída}            &  Backlog Atualizado\\
                    \hline
                    \textbf{Responsável}            & Product Owner e  Gerente do Produto\\
                    \hline
                \end{tabular}
            \end{table}

              \begin{table}[H]
                \centering
                \caption{Elaborar PI(Program Increment)}
                \begin{tabular}{c|p{10cm}}
                    \hline
                    \textbf{Nome}            & Elaborar PI(Program Increment)\\
                    \hline
                    \textbf{Identificador} & PR09\\
                    \hline
                    \textbf{Descrição}   & Definir quais serão os incrementos do produto de acordo com as necessidades do cliente\\
                    \hline
                    \textbf{Entrada}           & Visão e Backlog de Features\\
                    \hline
                    \textbf{Saída}            &  Incremento de Software\\
                    \hline
                    \textbf{Responsável}            & Gerente do Produto, Arquiteto do Sistema, PO, Scrum Master\\
                    \hline
                \end{tabular}
            \end{table}

     \large{Atividades do Time\\}

              \begin{table}[H]
                \centering
                \caption{Definir Histórias do Usuário}
                \begin{tabular}{c|p{10cm}}
                    \hline
                    \textbf{Nome}            & Definir Histórias do Usuário\\
                    \hline
                    \textbf{Identificador} & T01\\
                    \hline
                    \textbf{Descrição}   & Fragmentar e refinar Features para obtenção das histórias do usuário
\\
                    \hline
                    \textbf{Entrada}           & N/A\\
                    \hline
                    \textbf{Saída}            &  Backlog do Time\\
                    \hline
                    \textbf{Responsável}            & Scrum master  e  PO\\
                    \hline
                \end{tabular}
            \end{table}

              \begin{table}[H]
                \centering
                \caption{Gerenciar Requisitos}
                \begin{tabular}{c|p{10cm}}
                    \hline
                    \textbf{Nome}            & Gerenciar Requisitos\\
                    \hline
                    \textbf{Identificador} & T02\\
                    \hline
                    \textbf{Descrição}   & Manter os requisitos\\
                    \hline
                    \textbf{Entrada}           & Requisitos\\
                    \hline
                    \textbf{Saída}            &  Requisitos atualizados\\
                    \hline
                    \textbf{Responsável}            & Scrum Master e time \\
                    \hline
                \end{tabular}
            \end{table}


             \begin{table}[H]
                \centering
                \caption{Priorizar Histórias do Usuário}
                \begin{tabular}{c|p{10cm}}
                    \hline
                    \textbf{Nome}            & Priorizar Histórias do Usuário\\
                    \hline
                    \textbf{Identificador} & T03\\
                    \hline
                    \textbf{Descrição}   & Definir quais histórias do usuário devem possuir maior prioridade sobre os outros\\
                    \hline
                    \textbf{Entrada}           & Backlog do Time\\
                    \hline
                    \textbf{Saída}            &  Backlog atualizado\\
                    \hline
                    \textbf{Responsável}            &  Scrum Master, time e Product Owner\\
                    \hline
                \end{tabular}
            \end{table}

             \begin{table}[H]
                \centering
                \caption{Sprint Planning }
                \begin{tabular}{c|p{10cm}}
                    \hline
                    \textbf{Nome}            & Sprint Planning \\
                    \hline
                    \textbf{Identificador} & T04\\
                    \hline
                    \textbf{Descrição}   & Realizar o planejamento da iteração\\
                    \hline
                    \textbf{Entrada}           & N/A\\
                    \hline
                    \textbf{Saída}            & Sprint Backlog\\
                    \hline
                    \textbf{Responsável}            & Scrum Master e  time \\
                    \hline
                \end{tabular}
            \end{table}

             \begin{table}[H]
                \centering
                \caption{Implementar Código}
                \begin{tabular}{c|p{10cm}}
                    \hline
                    \textbf{Nome}            & Implementar Código\\
                    \hline
                    \textbf{Identificador} & T05\\
                    \hline
                    \textbf{Descrição}   & Implementação das histórias do usuário\\
                    \hline
                    \textbf{Entrada}           & Sprint Backlog\\
                    \hline
                    \textbf{Saída}            &  Código fonte\\
                    \hline
                    \textbf{Responsável}            & Time\\
                    \hline
                \end{tabular}
            \end{table}

             \begin{table}[H]
                \centering
                \caption{Sprint Review}
                \begin{tabular}{c|p{10cm}}
                    \hline
                    \textbf{Nome}            & Sprint Review\\
                    \hline
                    \textbf{Identificador} & T06\\
                    \hline
                    \textbf{Descrição}   & Realizar o ritual Sprint review\\
                    \hline
                    \textbf{Entrada}           & Código Fonte\\
                    \hline
                    \textbf{Saída}            &  Código fonte validado\\
                    \hline
                    \textbf{Responsável}            & Scrum Master, time e PO\\
                    \hline
                \end{tabular}
            \end{table}

             \begin{table}[H]
                \centering
                \caption{Sprint Retrospective}
                \begin{tabular}{c|p{10cm}}
                    \hline
                    \textbf{Nome}            & Sprint Retrospective\\
                    \hline
                    \textbf{Identificador} & T07\\
                    \hline
                    \textbf{Descrição}   & Realizar o ritual Sprint Retrospective\\
                    \hline
                    \textbf{Entrada}           & Propostas de Melhoria\\
                    \hline
                    \textbf{Saída}            &  Melhorias\\
                    \hline
                    \textbf{Responsável}            & Scrum Master e  time \\
                    \hline
                \end{tabular}
            \end{table}
Ac
}
