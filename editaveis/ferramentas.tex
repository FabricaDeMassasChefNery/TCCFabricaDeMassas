\chapter[Ferramentas de Gestão de Requisitos]{Ferramentas de Gestão de Requisitos}

\tab Foram realizadas pesquisas comparativas entre as ferramentas para gerir os requisitos, podendo elas serem versões Web ou ainda em versão de Desktop. Foram escolhidas então: Jira, Innoslate e Rally Dev.\\

\section {\large{Critérios}}

\tab Para esta análise, foram listadas 5 características julgadas importantes para a gestão de requisitos para o projeto, referentes à Usabilidade, Rastreabilidade, Gestão de Mudanças, Flexibilidade e Licença.\\
\tab \textbf{Usabilidade:} Analisa se a usabilidade da ferramenta é de fato intuitiva ou não, o que pode vir a trazer contratempos para a equipe.\\
\tab \textbf{Licença:} Analisa a licença da ferramenta, caso seja grátis, paga, se possui isenção para projetos educativos, etc.\\
\tab \textbf{Rastreabilidade:} Analisa se é possível rastrear de forma eficaz a ferramenta.\\
\tab \textbf{Gestão de Mudanças:} Analisa quais são as funcionalidades que ajudam na gestão de mudanças, análise de impacto, e escopo.\\
\tab \textbf{Flexibilidade:} Analisa a flexibilidade de personalização da ferramenta ao contexto do projeto.\\

\section {\large{Pontuação}}

\tab Para conseguir escolher com exatidão, foi atribuído uma pontuação a tais características com o intuito de esclarecer numericamente qual a ferramenta que nos auxiliaria. A pontuação foi ponderada de 0 à 5 sendo mais próximo de 5 muito pertinente às necessidades do projeto, e mais próximo de 0 muito divergente do fim das reais necessidades do projeto.\\

\section {\large{Resultados}}

\begin{table}[h]
\centering
\vspace{0.5cm}
\begin{tabular}{ c | c | c | c } \hline
Item/Ferramenta & Atlasian Jira  & Innoslate & Rally Dev \\ \hline
Usabilidade & 4 & 3 & 3 \\ \hline
Licença & 3 & 4 & 4 \\ \hline
Rastreabilidade & 1 & 5 & 4 \\ \hline
Gestão de mudanças & 1 & 4 & 4 \\ \hline
Flexibilidade & 2 & 3 & 5  \\ \hline
TOTAL & 11 & 19 & 20 \\ \hline
\end{tabular}
\caption{Pontuação das ferramentas}
\end{table}

\section {\large{Descrição das notas atribuídas}}

\textbf{Usabilidade:}\\
\tab Atlassian Jira: Interface bonita; Nomes bem intuitivos; Plugins gráficos interessantes.\\
\tab Innoslate: Interface satisfatória; \\
\tab RallyDev: Interface satisfatória; Ícones intuitivos;\\

\textbf{Licença:}\\
\tab Atlassian Jira: Gratuito por 30 dias;\\
\tab Innoslate: Versão grátis, com restrições;\\
\tab RallyDev: Versão grátis, com restrições;\\

\textbf{Rastreabilidade:}\\
\tab Atlassian Jira : Hierarquia pouco intuitiva; não existe ferramenta visual para controlar importância de requisitos em relação aos demais.\\
\tab Innoslate: Rastreabilidade bastante intuitiva; Geração automática de índice de qualidade de requisitos e numeração na criação de entidade.\\
\tab RallyDev: Hierarquia com fácil localização; Opção de linkar requisitos filhos;\\

\textbf{Gestão de mudanças:}\\
\tab Atlassian Jira: Não aparenta ter algum feedback de mudanças do próprio autor ou de outros autores;\\
\tab Innoslate: Controle de versão eficaz; Notificações com últimas alterações com autor e data;\\
\tab RallyDev: Controle de versão eficaz;\\

\textbf{Flexibilidade:} \\
\tab Atlassian Jira: Não apresenta ter opção de personalização;\\
\tab Innoslate: Opções de fixar e desafixar abas importantes para o projeto;\\
\tab RallyDev: Totalmente flexivel para modificações relevantes para o projeto com inúmeras possibilidades de plugins.\\

\section {\large{Escolha da ferramenta}}

\tab Após uma profunda análise em cada uma das ferramentas estudadas, foi decidido que o Rally Dev será utilizada para o gerenciamento de requisitos. Suas características de personalização e controle de versão foram fundamentais para a criação do próprio modelo de rastreabilidade do projeto.  \\
